\pdfminorversion=7%
\documentclass[aspectratio=169,mathserif,notheorems]{beamer}%
%
\xdef\bookbaseDir{../../bookbase}%
\xdef\sharedDir{../../shared}%
\RequirePackage{\bookbaseDir/styles/slides}%
\RequirePackage{\sharedDir/styles/styles}%
%
%
\title{An Introduction to Optimization}%
%
%
\begin{document}%
\startPresentation%
%
\section{Introduction}%
%
\begin{frame}%
\frametitle{What is Optimization?}%
\begin{itemize}%
\item Are are \alert{three} ways to approach this topic\only<-1>{.}\uncover<2->{\medskip%
\begin{enumerate}%
%
\item Optimization is an extension of school mathematics into a field where equations are no longer enough and exact solutions \alert{cannot always be reached}.\medskip%
%
\item<3-> Optimization is the art of solving \alert{hard} problems.\medskip%
%
\item<4-> Optimization means searching for \alert{superlatives}.%
\end{enumerate}}%
\end{itemize}%
%
\end{frame}%
%
\xdef\sharedSlideTitle{Optimization = Search for Superlatives}%
%
\begin{frame}[t]%
\expandafter\frametitle{\sharedSlideTitle}%
%
\locateGraphic{1}{width=0.85\paperwidth}{\sharedDir/graphics/optimization_superlatives/optimization_superlatives}{0.075}{0.18}%
\locate{1}{\parbox{0.33\paperwidth}{\noindent\footnotesize%
The \emph{superlative} form of an adjective is used to show that something has a quality to the greatest or least degree.%
}}{0.65}{0.86}%
%
\locateGraphic{2}{width=0.35\paperwidth}{\sharedDir/graphics/optimization_superlatives/optimization_superlatives_shortest}{0.63}{0.01}%
\locateGraphic{3}{width=0.35\paperwidth}{\sharedDir/graphics/optimization_superlatives/optimization_superlatives_least}{0.63}{0.01}%
\locateGraphic{4}{width=0.35\paperwidth}{\sharedDir/graphics/optimization_superlatives/optimization_superlatives_fewest}{0.63}{0.01}%
\locateGraphic{5}{width=0.35\paperwidth}{\sharedDir/graphics/optimization_superlatives/optimization_superlatives_earliest}{0.63}{0.01}%
\locateGraphic{6}{width=0.35\paperwidth}{\sharedDir/graphics/optimization_superlatives/optimization_superlatives_longest}{0.63}{0.01}%
\locateGraphic{7}{width=0.35\paperwidth}{\sharedDir/graphics/optimization_superlatives/optimization_superlatives}{0.63}{0.01}%
%
\locateGraphic{2}{width=0.4\paperwidth}{\sharedDir/graphics/hfuu_campus_map/hfuu_campus_map}{0.1}{0.27}%
\locateGraphic{2}{width=0.4\paperwidth}{\sharedDir/graphics/hfuu_campus_map/hfuu_campus_map_01_start_goal}{0.1}{0.27}%
%
\locateGraphic{3}{width=0.5\paperwidth}{\sharedDir/graphics/transport/transport}{0.35}{0.33}%
%
\locateGraphic{4}{width=0.8\paperwidth}{\sharedDir/graphics/bin_packing/bin_packing}{0.1}{0.47}%
%
\locateGraphic{5}{width=0.42\paperwidth}{\sharedDir/graphics/jssp_production_diagram/jssp_production_diagram}{0.29}{0.516}%
%
\locateGraphic{6}{width=0.5\paperwidth}{\sharedDir/graphics/wireless_network/wireless_network}{0.62}{0.32}%
%
\locate{}{\parbox{0.95\paperwidth}{\noindent%
\begin{itemize}%
\item Optimization means finding superlatives.%
%
\item<2-> Find the \alert{shortest} path from start to goal.%
%
\item<3-> Pick up and deliver packages from different places\\to customers using the \alert{least} amount of fuel.%
%
\item<4-> Pack a set of things into the \alert{fewest} boxes.%
%
\item<5-> Assign tasks to machines such that we can finish our work the at \alert{earliest}\only<-5>{ }\only<6->{\\}possible time.%
%
\item<6-> Find a strategy to manage the power of the nodes in this sensor\\network so that full coverage is guaranteed for the \alert{longest}\\possible duration.%
%
\item<7-> And so on.%
\end{itemize}%
}}{0.025}{0.0875}%
%
\end{frame}%
%
%
%
\begin{frame}%
\frametitle{Business}%
%
\locateGraphic{1}{width=1.05\paperwidth}{\sharedDir/graphics/five_aspects/five_aspects_1}{-0.1}{0.05}%
\locateGraphic{2}{width=1.05\paperwidth}{\sharedDir/graphics/five_aspects/five_aspects_2}{-0.1}{0.05}%
\locateGraphic{3}{width=1.05\paperwidth}{\sharedDir/graphics/five_aspects/five_aspects_3}{-0.1}{0.05}%
\locateGraphic{4}{width=1.05\paperwidth}{\sharedDir/graphics/five_aspects/five_aspects_4}{-0.1}{0.05}%
\locateGraphic{5}{width=1.05\paperwidth}{\sharedDir/graphics/five_aspects/five_aspects_5}{-0.1}{0.05}%
\locateGraphic{6}{width=1.05\paperwidth}{\sharedDir/graphics/five_aspects/five_aspects_6}{-0.1}{0.05}%
\locateGraphic{7}{width=1.05\paperwidth}{\sharedDir/graphics/five_aspects/five_aspects}{-0.1}{0.05}%
%
\end{frame}%
%
\begin{frame}%
\frametitle{Optimization}%
\begin{itemize}%
%
\item There are incredibly many problems from a very wide area where we can use optimization and \glsFull{fieldOR}.\medskip%
%
\item<2-> How is this related to what you already learned?%
\end{itemize}%
\end{frame}%
%
\section{Problems that we can Solve with Equations}%
%
\definecolor{square-color}{HTML}{A03214}%
\definecolor{rectangle-color}{HTML}{0000A0}%
\definecolor{small-triangle-color}{HTML}{AA0078}%
\definecolor{large-triangle-color}{HTML}{008200}%
\definecolor{x-color}{HTML}{00788C}%
%
\protected\gdef\aRectangle{{\color{rectangle-color}{\ensuremath{A_{\text{\ding{122}}}}}}}%
\protected\gdef\aRectangleDiff{{\color{rectangle-color}{\ensuremath{A'_{\text{\ding{122}}}}}}}%
\protected\gdef\aRectangleMax{{\color{rectangle-color}{\ensuremath{\widehat{A_{\text{\ding{122}}}}}}}}%
\protected\gdef\aSquare{{\color{square-color}{\ensuremath{A_{\text{\ding{110}}}}}}}%
\protected\gdef\aSmallTriangle{{\color{small-triangle-color}{\ensuremath{A_{\text{\ding{116}}}}}}}%
\protected\gdef\aLargeTriangle{{\color{large-triangle-color}{\ensuremath{A_{\text{\ding{115}}}}}}}%
\protected\gdef\aX{{\color{x-color}{\ensuremath{x}}}}%
\protected\gdef\aXMax{{\color{x-color}{\ensuremath{\hat{x}}}}}%
\protected\gdef\aRectangleOfX{\ensuremath{\aRectangle(\aX)}}%
\protected\gdef\aRectangleOfXDiff{\ensuremath{\aRectangleDiff(\aX)}}%
%
\begin{frame}[t]%
\frametitle{A Problem that we can Solve with (High School Maths) Equations}%
\begin{itemize}%
\item A \textcolor{rectangle-color}{rectangle} should be placed inside a \textcolor{square-color}{square} with side\\length~13cm as sketched.%
%
\item<2-> What area has the \alert{largest} such \textcolor{rectangle-color}{rectangle}?%
\end{itemize}%
%
\uncover<3->{\parbox{0.6\paperwidth}{\noindent%
\begin{align}\nonumber%
%
\only<-11>{%
\aRectangle&=\aSquare-2\aSmallTriangle-2\aLargeTriangle%
}%
%
\only<-11>{%
\nonumber\uncover<4->{\nonumber\\\nonumber%
\aRectangleOfX&={\color{square-color}{\left(13\text{cm}^2\right)}}-2{\color{small-triangle-color}{\left(\tfrac{1}{2}\aX^2\right)}}-2{\color{large-triangle-color}{\left(\tfrac{1}{2}(13\text{cm}-\aX)^2\right)}}%
}%
}%
%
\only<-11>{%
\nonumber\uncover<5->{\nonumber\\\nonumber%
\aRectangleOfX&=169\text{cm}^2-\aX^2-(13\text{cm}-\aX)^2%
}%
%
\nonumber\uncover<6->{\nonumber\\\nonumber%
\aRectangleOfX&=169\text{cm}^2-\aX^2-169\text{cm}^2+26\text{cm}*\aX-\aX^2%
}}%
%
\nonumber\uncover<7->{\only<-11>{\nonumber\\}\nonumber%
\aRectangleOfX&=-2\aX^2+26\text{cm}*\aX%
}%
%
\only<-11>{%
\nonumber\uncover<8->{\nonumber\\\nonumber%
\aRectangleOfXDiff&=-4\aX+26\text{cm}%
}%
%
\nonumber\uncover<9->{\nonumber\\\nonumber%
0&=-4\aXMax+26\text{cm}\uncover%
}%
%
\nonumber\uncover<10->{\nonumber\\\nonumber%
4\aXMax&=26\text{cm}%
}%
}%
%
\nonumber\uncover<11->{\nonumber\\\nonumber%
\aXMax&=6.5\text{cm}%
}%
%
\nonumber\uncover<13->{\nonumber\\\nonumber%
\aRectangleMax&=\aRectangle(\aXMax)=\aRectangle(6.5\text{cm})%
}%
%
\nonumber\uncover<13->{\nonumber\\\nonumber%
\aRectangleMax&=-2(6.5\text{cm})^2+26\text{cm}*6.5\text{cm}%
}%
%
\nonumber\uncover<14->{\nonumber\\\nonumber%
\scalebox{1.5}{\ensuremath{\aRectangleMax\;}}&\scalebox{1.5}{\ensuremath{=\;84.5\text{cm}^2}}%
}%
%
\nonumber%
\end{align}%
}}%
%
\uncover<15->{%
\begin{center}%
\scalebox{1.5}{\textbf{\Large{\alert{Solved.}}}}%
\end{center}%
}%%
%
\strut\\[\paperheight]\strut%
%
\locateGraphic{1-2}{width=0.3\paperwidth}{\sharedDir/graphics/rectangle_problem/rectangle_problem_1}{0.67}{0.15}%
\locateGraphic{3}{width=0.3\paperwidth}{\sharedDir/graphics/rectangle_problem/rectangle_problem_2}{0.67}{0.15}%
\locateGraphic{4-}{width=0.3\paperwidth}{\sharedDir/graphics/rectangle_problem/rectangle_problem_3}{0.67}{0.15}%
%
\end{frame}%
%
\begin{frame}%
\frametitle{Problems that we can solve with Equations}%
\begin{itemize}%
\item We can actually solve a lot of problems.%
%
\item<2-> We just used an equation.%
%
\item<3-> We did computations in multiple steps.%
%
\item<4-> Regardless how the previous problem would be parameterized~(say, 16cm instead of 13cm), we could perform the exactly same steps.%
%
\item<5-> \alert{Can we always do that?}%
\end{itemize}%
\end{frame}%
%
\section{Problems that we can Solve with an Algorithm}%
%
\begin{frame}%
\frametitle{Problems that we can Solve with Algorithms}%
\begin{itemize}%
%
\item \alert<-1>{Can we always do that?}%
%
\item<2-> \alert{Can we always solve problems with a pre-defined number of steps?}%
%
\item<3-> \alert{No.}%
%
\item<4-> There are problems that we cannot solve with equations, but with algorithms.%
%
\item<5-> And some problems require us to use algorithms which perform different numbers of steps of different inputs.%
\end{itemize}%
\end{frame}%
%
\definecolor{flip-color}{HTML}{0000A0}%
\definecolor{flop-color}{HTML}{AA0078}%
\protected\gdef\flipColor#1#2{{\color<#1>{flip-color}{#2}}}%
\protected\gdef\flopColor#1#2{{\color<#1>{flop-color}{#2}}}%
%
\begin{frame}[t]%
\frametitle{A Problem that we can Solve with (a High School) Algorithm}%
\begin{itemize}%
%
\item What is the greatest common divisor~($\operatorname{gcd}$) of~$a=257\decSep9731$ and~$b=164\decSep1647$?%
\end{itemize}%
%
\uncover<2->{\vspace*{-0.4em}%
\parbox{0.6\paperwidth}{\noindent%
\begin{itemize}%%
\only<-8>{%
\item The $\operatorname{gcd}$ can be computed with the Euclidean algorithm\cite{EHF2008EEOGTGOJLH11FEEEELIEILHIATBG11EAPWMETBFR:ENT,B1999FAOTBEA,TKY2016BEOEAOTCEG} by \citeauthor{EHF2008EEOGTGOJLH11FEEEELIEILHIATBG11EAPWMETBFR:ENT}, who lived about 300~\emph{\glsFull{BCE}}.%
}%
%
\only<-10>{%
\item<3-> The $\operatorname{gcd}$ of two positive natural numbers~$a\in\naturalNumbersO$ and~$b\in\naturalNumbersO$ is the largest number~$g\in\naturalNumbersO=\operatorname{gcd}(a,b)$ which divides both~$a$ and~$b$ without remainder.%
}%
%
\only<-11>{%
\item<4-> If $a=b$, then obviously $\operatorname{gcd}(a,b)=a=b$.%
%
\item<5-> Otherwise, we know that $a=ig$ for some~$i\in\naturalNumbersO$ and that $b=jg$ for some~$j\in\naturalNumbersO$.%
%
\item<6-> Without loss of generality, let's assume that~$a>b$.%
%
\item<7-> Then it holds that $c=a-b=(i-j)g$.%
%
\item<8-> Thus, $g$ also divides~$c$ without remainder, i.e., $\operatorname{gcd}(a,b)=\operatorname{gcd}(c,b)$.%
%
\item<9-> It must also be that $a-b<a$.%
%
\item<10-> We can replace~$a$ with~$a-b$.%
}%
%
\item<11-> We can repeatedly subtract the smaller from the larger number until we \inQuotes{converge.}%
\end{itemize}%
%
%
\uncover<12->{%
\begin{center}%
\resizebox{0.95\linewidth}{!}{%
\begin{tabular}{ccc}%
\hline%
\textbf{bigger number:~$a$}&\textbf{smaller number:~$b$}&\textbf{$a-b$}\\%
\hline%
%
\uncover<13->{%
257\decSep9731&164\decSep1647&\flipColor{13-14}{93\decSep8084}%
}%
%
\uncover<14->{\\%
164\decSep1647&\flipColor{14}{93\decSep8084}&\flopColor{14-15}{70\decSep3563}%
}%
%
\uncover<15->{\\%
93\decSep8084&\flopColor{15}{70\decSep3563}&\flipColor{15-16}{23\decSep4521}%
}%
%
\uncover<16->{\\%
70\decSep3563&\flipColor{16}{23\decSep4521}&\flopColor{16-17}{46\decSep9042}%
}%
%
\uncover<17->{\\%
\flopColor{17}{46\decSep9042}&\flipColor{17}{23\decSep4521}&\flipColor{17}{23\decSep4521}%
}%
%
\\\hline%
\end{tabular}%
}%
\end{center}%
%
\uncover<18->{\medskip\noindent%
\begin{itemize}%
\item $\operatorname{gcd}(257\decSep9731,164\decSep1647)=23\decSep4521$.%
\end{itemize}%
\uncover<19->{\begin{center}%
\scalebox{1.5}{\textbf{\Large{\alert{Solved.}}}}%
\end{center}}%
}%
}}}%
%
\strut\vspace{\paperheight}\strut%
%
\locateGraphicTB[\parbox{0.6\paperwidth}{\noindent\vspace*{-0.5em}\tiny{%
An illustration of Euclid of Alexandria, \href{https://www.antike-griechische.de/Euklid.pdf}{attributed to} Charles Paul Landon~(1760--1826). Source:~\href{https://fr.vikidia.org/wiki/Cat\%C3\%A9gorie:Image_Euclide}{Vikidia}, where it is listed as \emph{domaine public.}}}%
]{2-}{width=0.32\paperwidth}{\sharedDir/graphics/euclidOfAlexandria/euclidOfAlexandria}{0.66}{0.25}%
\end{frame}%
%
\begin{frame}%
\frametitle{Euclidean Algorithm}%
\begin{itemize}%
%
\item The number of steps that the algorithm needs depends on the input.%
%
\item<2-> The Euclidean Algorithm can be implemented more \alert<4>{efficiently} using \inQuotes{division remainders} instead of subtractions.%
%
\item<3-> It can be made even more \alert<4>{efficient} using a binary variant {\dots} that already existed in China in the first century \glsFull{CE}\cite{B1999FAOTBEA}, published in the famous \emph{Jiu Zhang Suanshu}~(九章算术)\cite{OR2003LH,SCL1999TNCOTMACAC,S1998LHATFGAOCM,D2010AALHOCAS,C2002LFLHADWTDM}.%
%
\item<4-> What does \alert<4>{efficient} even mean?%
%
\item<5-> \alert{efficient = fast\uncover<6->{ and does not need much memory}}%
%
\item<7-> (Side note:~The binary Euclidean algorithm can be computed in~$c*(\log{a}+\log{b})$ steps where $c>0$~is some constance. It needs two memory cells\cite{B1999FAOTBEA,BB1987ASAFEGC}.)%
\end{itemize}%
\end{frame}%
%
\section{Problems that Algorithms can Solve Fast and Efficiently}%
%
\definecolor{start-color}{HTML}{FF00C8}%
\definecolor{goal-color}{HTML}{00C0C0}%
\definecolor{path-red}{HTML}{FF0000}%
\definecolor{path-pink}{HTML}{FF00FF}%
\definecolor{path-green}{HTML}{008000}%
\definecolor{path-cyan}{HTML}{00C8C8}%
\definecolor{path-gray}{HTML}{646464}%
\protected\gdef\apath#1#2{{\color{path-#1}{#2}}}%
\protected\gdef\noapath{\protected\gdef\apath##1##2{##2}}%
%
\begin{frame}[t]%
\frametitle{Find the Shortest Path from {\color{start-color}{Start}} to {\color{goal-color}{Goal}}}%
%
\locate{2-}{\parbox{0.47\paperwidth}{\noindent%
\begin{itemize}%
%
\item I am at the {\color{start-color}{starting point 合肥大学南二区北大门}}.%
%
\item<3-> I want to go to the {\color{goal-color}{goal:~食堂}}.
%
\only<-8>{%
\item<4-> How do I get there the \alert{fastest}?%
%
\item<5-> We know the campus map.%
%
\item<6-> We want to compute the shortest path~(before actually walking it).%
%
\item<7-> We know all the intersections where I could make turns.%
}%
%
\only<-12>{%
\item<9-> For example, I could walk for \apath{red}{$27s$} to this intersection.%
%
\item<10-> Or for \apath{red}{$83s$} to that one.%
%
\item<11-> Or for \apath{red}{$195s$} to that one.%
%
\item<12-> Which one should we pick?%
}%
%
\only<-21>{%
\item<13-> We have 3~choices: (a)~$\apath{red}{27s}\only<17->{+\apath{green}{213s}}$, (b)~$\apath{red}{83s}\only<19->{+\apath{green}{221s}}$, (c)~$\apath{red}{195s}\only<21->{+\apath{green}{362s}}$.%
%
\item<14-> For each intersection, we can compute the airline distance~(\apath{green}{as the crow flies}) to the~{\color{goal-color}{食堂}}.%
%
\item<15-> The actual walking distance can never be shorter than that.%
}%
%
\only<-25>{%
\item<22-> We have 3~choices: (a)~$240s=\apath{red}{27s}+\apath{green}{213s}$, (b)~$\only<23->{304s=}\apath{red}{83s}+\apath{green}{221s}$, (c)~$\only<24->{557s=}\apath{red}{195s}+\apath{green}{362s}$.%
}%
%
\only<-33>{%
\item<25-> The most interesting candidate for the first step is clearly~(a).%
%
\only<-29>{%
\item<26-> We have 2~remaining unexplored choices: (b)~$304s=\apath{red}{83s}+\apath{green}{221s}$, (c)~$557s=\apath{red}{195s}+\apath{green}{362s}$.%
}%
%
\item<27-> From~(a), we can go to~(d) by walking for~\apath{red}{$126s$}\only<28->{ or to~(e) by walking for~\apath{red}{$59s$}}.%
%
\item<29-> For both we compute the \apath{green}{distance as the crow flies}.%
}%
%
\only<-34>{%
\item<31-> We \only<-31>{had 2~remaining }\only<32>{now have 3~}\only<33->{now have 4~}unexplored choices: (b)~$304s=\apath{red}{83s}+\apath{green}{221s}$, (c)~$557s=\apath{red}{195s}+\apath{green}{362s}$\only<32->{, (d)~$273s=\apath{red}{153s}+\apath{green}{120s}$}\only<33->{, (e)~$256s=\apath{red}{86s}+\apath{green}{170s}$}.%
}%
%
\only<-35>{%
\item<34-> (e)~it is\dots%
}%
%
\only<-37>{%
\item<35-> We have 3~remaining unexplored choices: (b)~$304s=\apath{red}{83s}+\apath{green}{221s}$, (c)~$557s=\apath{red}{195s}+\apath{green}{362s}$, (d)~$273s=\apath{red}{153s}+\apath{green}{120s}$.%
}%
%
\only<-38>{%
\item<36-> From this point~(e), we only have one reasonable choice where to go next.%
}%
%
\only<-48>{%
\item<37-> And we again compute the {\apath{green}{airline distance}} to the~{\color{goal-color}{goal}}.%
%
\only<-39>{%
\item<38-> We now have 4~choices: (b)~$304s=\apath{red}{83s}+\apath{green}{221s}$, (c)~$557s=\apath{red}{195s}+\apath{green}{362s}$, (d)~$273s=\apath{red}{153s}+\apath{green}{120s}$, (f)~$262s=\apath{red}{132s}+\apath{green}{130s}$.%
}%
%
\only<-47>{%
\item<39-> (f)~it is\dots%
}%
%
\only<-45>{%
\item<40-> We have 3~remaining unexplored choices: (b)~$304s=\apath{red}{83s}+\apath{green}{221s}$, (c)~$557s=\apath{red}{195s}+\apath{green}{362s}$, (d)~$273s=\apath{red}{153s}+\apath{green}{120s}$.%
}%
%
\item<41-> From (f), we have two possible choices to continue.%
%
\item<46-> We now have \only<-46>{4}\only<47->{5}~unexplored choices: (b)~$304s=\apath{red}{83s}+\apath{green}{221s}$, (c)~$557s=\apath{red}{195s}+\apath{green}{362s}$, (d)~$273s=\apath{red}{153s}+\apath{green}{120s}$, (g)~$262s=\apath{red}{160s}+\apath{green}{102s}$\only<47->{, (h)~$316s=\apath{red}{175s}+\apath{green}{141s}$}.%
}%
%
\only<-54>{%
\item<48-> (g)~it is\dots%
%
\item<49-> We now have 4~unexplored choices: (b)~$304s=\apath{red}{83s}+\apath{green}{221s}$, (c)~$557s=\apath{red}{195s}+\apath{green}{362s}$, (d)~$273s=\apath{red}{153s}+\apath{green}{120s}$, (h)~$316s=\apath{red}{175s}+\apath{green}{141s}$.%
}%
%
\only<-57>{%
\item<50-> And we get two new choices.%
%
\item<55-> We now have \only<-55>{5}\only<56->{6}~unexplored choices: (b)~$304s=\apath{red}{83s}+\apath{green}{221s}$, (c)~$557s=\apath{red}{195s}+\apath{green}{362s}$, (d)~$273s=\apath{red}{153s}+\apath{green}{120s}$, (h)~$316s=\apath{red}{175s}+\apath{green}{141s}$, (i)~$265s=\apath{red}{215s}+\apath{green}{50s}$\only<56->{, (j)~$277s=\apath{red}{190s}+\apath{green}{87s}$}.%
%
\item<57-> (i)~it is\dots%
}%
%
\only<-63>{%
\item<58-> We now have 5~unexplored choices: (b)~$304s=\apath{red}{83s}+\apath{green}{221s}$, (c)~$557s=\apath{red}{195s}+\apath{green}{362s}$, (d)~$273s=\apath{red}{153s}+\apath{green}{120s}$, (h)~$316s=\apath{red}{175s}+\apath{green}{141s}$, (j)~$277s=\apath{red}{190s}+\apath{green}{87s}$.%
}%
%
\only<-66>{%
\item<59-> From~(i), we again got two new choices.%
}%
%
\only<70->{\noapath}%
\item<64-> We now have \only<75->{0}\only<74>{1}\only<73>{2}\only<72>{3}\only<71>{4}\only<70>{5}\only<-64,67-70>{6}\only<65-66>{7}~unexplored choices: {\color<72->{path-gray}{(b)~$304s=\apath{red}{83s}+\apath{green}{221s}$}}, {\color<75->{path-gray}{(c)~$557s=\apath{red}{195s}+\apath{green}{362s}$}}, {\color<70->{path-gray}{(d)~$273s=\apath{red}{153s}+\apath{green}{120s}$}}, {\color<73->{path-gray}{(h)~$316s=\apath{red}{175s}+\apath{green}{141s}$}}, {\color<71->{path-gray}{(j)~$277s=\apath{red}{190s}+\apath{green}{87s}$}}, {\color<74->{path-gray}{(k)~$316s=\apath{red}{252s}+\apath{green}{64s}$}}\only<65-66>{, (l)~$268s=\apath{red}{236s}+\apath{green}{32s}$}.%
%
\only<-69>{%
\only<-68>{%
\item<66-> (l)~it is\dots%
}%
%
\item<67-> There is only one choice to continue:~go to the {\color{goal-color}{goal}}!%
%
\item<68-> We found a first complete path from {\color{start-color}{start}} to {\color{goal-color}{goal}}.%
%
\item<69-> It has the total length~\apath{red}{$268s$}.%
}%
%
\only<-70>{%
\item<70-> (d)~definitely will need more than~$268s$.%
}%
%
\only<-71>{%
\item<71-> (j)~definitely will need more than~$268s$.%
}%
%
\only<-72>{%
\item<72-> (b)~definitely will need more than~$268s$.%
}%
%
\only<-73>{%
\item<73-> (h)~definitely will need more than~$268s$.%
}%
%
\only<-74>{%
\item<74-> (k)~definitely will need more than~$268s$.%
}%
%
\only<-75>{%
\item<75-> (c)~definitely will need more than~$268s$.%
}%
%
\item<76-> We found the shortest possible path and it takes~$268s$.%
%
\item<77-> \scalebox{1.5}{\textbf{\Large{\alert{Solved.}}}}%
%
\end{itemize}%
}}{0.51}{0.1}%075}%
%
\locateGraphic{}{width=0.545\paperwidth}{\sharedDir/graphics/hfuu_campus_map_astar/hfuu_campus_map_astar}{-0.02}{0.15}%
%
\locateGraphic{2-6}{width=0.545\paperwidth}{\sharedDir/graphics/hfuu_campus_map_astar/hfuu_campus_map_astar_01_start_goal}{-0.02}{0.15}%
\locateGraphic{8-}{width=0.55045\paperwidth}{\sharedDir/graphics/hfuu_campus_map_astar/hfuu_campus_map_astar_03_overlay}{-0.02421}{0.1449}%
%
\locateGraphic{7-8}{width=0.545\paperwidth}{\sharedDir/graphics/hfuu_campus_map_astar/hfuu_campus_map_astar_02_paths}{-0.02}{0.15}%
%
\locateGraphic{9}{width=0.545\paperwidth}{\sharedDir/graphics/hfuu_campus_map_astar/hfuu_campus_map_astar_04a}{-0.02}{0.15}%
\locateGraphic{10}{width=0.545\paperwidth}{\sharedDir/graphics/hfuu_campus_map_astar/hfuu_campus_map_astar_04b}{-0.02}{0.15}%
\locateGraphic{11-15}{width=0.545\paperwidth}{\sharedDir/graphics/hfuu_campus_map_astar/hfuu_campus_map_astar_04c}{-0.02}{0.15}%
\locateGraphic{16-17}{width=0.545\paperwidth}{\sharedDir/graphics/hfuu_campus_map_astar/hfuu_campus_map_astar_04d}{-0.02}{0.15}%
\locateGraphic{18-19}{width=0.545\paperwidth}{\sharedDir/graphics/hfuu_campus_map_astar/hfuu_campus_map_astar_04e}{-0.02}{0.15}%
\locateGraphic{20-21}{width=0.545\paperwidth}{\sharedDir/graphics/hfuu_campus_map_astar/hfuu_campus_map_astar_04f}{-0.02}{0.15}%
%
\locateGraphic{22}{width=0.545\paperwidth}{\sharedDir/graphics/hfuu_campus_map_astar/hfuu_campus_map_astar_04g}{-0.02}{0.15}%
\locateGraphic{23}{width=0.545\paperwidth}{\sharedDir/graphics/hfuu_campus_map_astar/hfuu_campus_map_astar_04h}{-0.02}{0.15}%
\locateGraphic{24}{width=0.545\paperwidth}{\sharedDir/graphics/hfuu_campus_map_astar/hfuu_campus_map_astar_04i}{-0.02}{0.15}%
%
\locateGraphic{25-26}{width=0.545\paperwidth}{\sharedDir/graphics/hfuu_campus_map_astar/hfuu_campus_map_astar_05}{-0.02}{0.15}%
%
\locateGraphic{27}{width=0.545\paperwidth}{\sharedDir/graphics/hfuu_campus_map_astar/hfuu_campus_map_astar_06a}{-0.02}{0.15}%
\locateGraphic{28}{width=0.545\paperwidth}{\sharedDir/graphics/hfuu_campus_map_astar/hfuu_campus_map_astar_06b}{-0.02}{0.15}%
%
\locateGraphic{29}{width=0.545\paperwidth}{\sharedDir/graphics/hfuu_campus_map_astar/hfuu_campus_map_astar_06c}{-0.02}{0.15}%
\locateGraphic{30-31}{width=0.545\paperwidth}{\sharedDir/graphics/hfuu_campus_map_astar/hfuu_campus_map_astar_06d}{-0.02}{0.15}%
%
\locateGraphic{32}{width=0.545\paperwidth}{\sharedDir/graphics/hfuu_campus_map_astar/hfuu_campus_map_astar_07a}{-0.02}{0.15}%
\locateGraphic{33}{width=0.545\paperwidth}{\sharedDir/graphics/hfuu_campus_map_astar/hfuu_campus_map_astar_07b}{-0.02}{0.15}%
%
\locateGraphic{34-35}{width=0.545\paperwidth}{\sharedDir/graphics/hfuu_campus_map_astar/hfuu_campus_map_astar_07c}{-0.02}{0.15}%
%
\locateGraphic{36}{width=0.545\paperwidth}{\sharedDir/graphics/hfuu_campus_map_astar/hfuu_campus_map_astar_08a}{-0.02}{0.15}%
\locateGraphic{37}{width=0.545\paperwidth}{\sharedDir/graphics/hfuu_campus_map_astar/hfuu_campus_map_astar_08b}{-0.02}{0.15}%
\locateGraphic{38}{width=0.545\paperwidth}{\sharedDir/graphics/hfuu_campus_map_astar/hfuu_campus_map_astar_09a}{-0.02}{0.15}%
\locateGraphic{39-41}{width=0.545\paperwidth}{\sharedDir/graphics/hfuu_campus_map_astar/hfuu_campus_map_astar_09b}{-0.02}{0.15}%
%
\locateGraphic{42}{width=0.545\paperwidth}{\sharedDir/graphics/hfuu_campus_map_astar/hfuu_campus_map_astar_10a}{-0.02}{0.15}%
\locateGraphic{43}{width=0.545\paperwidth}{\sharedDir/graphics/hfuu_campus_map_astar/hfuu_campus_map_astar_10b}{-0.02}{0.15}%
\locateGraphic{44}{width=0.545\paperwidth}{\sharedDir/graphics/hfuu_campus_map_astar/hfuu_campus_map_astar_10c}{-0.02}{0.15}%
%
\locateGraphic{45}{width=0.545\paperwidth}{\sharedDir/graphics/hfuu_campus_map_astar/hfuu_campus_map_astar_10d}{-0.02}{0.15}%
\locateGraphic{46}{width=0.545\paperwidth}{\sharedDir/graphics/hfuu_campus_map_astar/hfuu_campus_map_astar_11a}{-0.02}{0.15}%
%
\locateGraphic{47}{width=0.545\paperwidth}{\sharedDir/graphics/hfuu_campus_map_astar/hfuu_campus_map_astar_11b}{-0.02}{0.15}%
%
\locateGraphic{48-50}{width=0.545\paperwidth}{\sharedDir/graphics/hfuu_campus_map_astar/hfuu_campus_map_astar_11c}{-0.02}{0.15}%
%
\locateGraphic{51}{width=0.545\paperwidth}{\sharedDir/graphics/hfuu_campus_map_astar/hfuu_campus_map_astar_12a}{-0.02}{0.15}%
\locateGraphic{52}{width=0.545\paperwidth}{\sharedDir/graphics/hfuu_campus_map_astar/hfuu_campus_map_astar_12b}{-0.02}{0.15}%
\locateGraphic{53}{width=0.545\paperwidth}{\sharedDir/graphics/hfuu_campus_map_astar/hfuu_campus_map_astar_12c}{-0.02}{0.15}%
\locateGraphic{54}{width=0.545\paperwidth}{\sharedDir/graphics/hfuu_campus_map_astar/hfuu_campus_map_astar_12d}{-0.02}{0.15}%
%
\locateGraphic{55}{width=0.545\paperwidth}{\sharedDir/graphics/hfuu_campus_map_astar/hfuu_campus_map_astar_13a}{-0.02}{0.15}%
\locateGraphic{56}{width=0.545\paperwidth}{\sharedDir/graphics/hfuu_campus_map_astar/hfuu_campus_map_astar_13b}{-0.02}{0.15}%
%
\locateGraphic{57-59}{width=0.545\paperwidth}{\sharedDir/graphics/hfuu_campus_map_astar/hfuu_campus_map_astar_13c}{-0.02}{0.15}%
%
\locateGraphic{60}{width=0.545\paperwidth}{\sharedDir/graphics/hfuu_campus_map_astar/hfuu_campus_map_astar_14a}{-0.02}{0.15}%
\locateGraphic{61}{width=0.545\paperwidth}{\sharedDir/graphics/hfuu_campus_map_astar/hfuu_campus_map_astar_14b}{-0.02}{0.15}%
\locateGraphic{62}{width=0.545\paperwidth}{\sharedDir/graphics/hfuu_campus_map_astar/hfuu_campus_map_astar_14c}{-0.02}{0.15}%
\locateGraphic{63}{width=0.545\paperwidth}{\sharedDir/graphics/hfuu_campus_map_astar/hfuu_campus_map_astar_14d}{-0.02}{0.15}%
%
\locateGraphic{64}{width=0.545\paperwidth}{\sharedDir/graphics/hfuu_campus_map_astar/hfuu_campus_map_astar_15a}{-0.02}{0.15}%
\locateGraphic{65}{width=0.545\paperwidth}{\sharedDir/graphics/hfuu_campus_map_astar/hfuu_campus_map_astar_15b}{-0.02}{0.15}%
\locateGraphic{66}{width=0.545\paperwidth}{\sharedDir/graphics/hfuu_campus_map_astar/hfuu_campus_map_astar_15c}{-0.02}{0.15}%
%
\locateGraphic{67-68}{width=0.545\paperwidth}{\sharedDir/graphics/hfuu_campus_map_astar/hfuu_campus_map_astar_16}{-0.02}{0.15}%
%
\locateGraphic{69}{width=0.545\paperwidth}{\sharedDir/graphics/hfuu_campus_map_astar/hfuu_campus_map_astar_17}{-0.02}{0.15}%
\locateGraphic{70}{width=0.545\paperwidth}{\sharedDir/graphics/hfuu_campus_map_astar/hfuu_campus_map_astar_18}{-0.02}{0.15}%
\locateGraphic{71}{width=0.545\paperwidth}{\sharedDir/graphics/hfuu_campus_map_astar/hfuu_campus_map_astar_19}{-0.02}{0.15}%
\locateGraphic{72}{width=0.545\paperwidth}{\sharedDir/graphics/hfuu_campus_map_astar/hfuu_campus_map_astar_20}{-0.02}{0.15}%
\locateGraphic{73}{width=0.545\paperwidth}{\sharedDir/graphics/hfuu_campus_map_astar/hfuu_campus_map_astar_21}{-0.02}{0.15}%
\locateGraphic{74}{width=0.545\paperwidth}{\sharedDir/graphics/hfuu_campus_map_astar/hfuu_campus_map_astar_22}{-0.02}{0.15}%
\locateGraphic{75-}{width=0.545\paperwidth}{\sharedDir/graphics/hfuu_campus_map_astar/hfuu_campus_map_astar_23}{-0.02}{0.15}%
%
\end{frame}%
%
\begin{frame}%
\frametitle{The \gls{algoAstar}for Finding Shortest Paths}%
\begin{itemize}%
%
\item We just applied the \pgls{algoAstar}\cite{HNR1968AFBFTHDOMCP,P1997AAP}.%
%
\item<2-> The \pgls{algoAstar} iteratively constructs a solution.%
%
\item<3-> It decides next step to test based on a combination of {\color{path-green}{heuristic}} and {\color{path-red}{cost}}.%
%
\item<4-> In the worst case, it needs to \inQuotes{look} at all intersections~(nodes) and streets~(edges) once.%
%
\item<5-> Usually, it is quite efficient to find shortest paths on maps.%
%
\item<6-> (If we look for shortest paths that do not visit any node twice, where the edge distances are not negative, where the number of choices per node is limited, and where the graph and path are not too big -- then this algorithm is very efficient.)%
%
\end{itemize}%
\end{frame}%
%
\section{Problems that Algorithms cannot Solve Efficiently \emph{and} Exactly}%
%
\begin{frame}%
\frametitle{Hard Problems~(1)}%
\begin{itemize}%
\item There exists a group of problems that are \emph{hard}.%
%
\item<2-> Actually, most of the problems I mention at the beginning of this talk belong to this group of hard problems.%
%
\item<3-> A hard problem cannot be solved both exactly and efficiently.%
%
\item<4-> Let‘s look at two quick examples.%
\end{itemize}%
\end{frame}%
%
\begin{frame}[t]%
\frametitle{Bin Packing}%
\begin{itemize}%
\item My car can carry $T\text{kg}$ of weight.%
%
\item<2-> I have $n$~objects, each with weight~$wi$ for~$i$ in~$1$ to~$n$.%
%
\item<3-> How can I pack my car so that I can carry them from~$A$ to $B$ with the fewest possible hauls?\cite{ZLWvdBTW2024RLSOT2RBPPWIR,ZWvdBTLTW2024RLSFTDBPAANRFFFA}%
\end{itemize}%
%
\locateGraphic{}{width=0.8\paperwidth}{\sharedDir/graphics/bin_packing/bin_packing}{0.1}{0.47}%
\end{frame}%
%
\begin{frame}[t]%
\frametitle{Traveling Salesperson Problem}%
\begin{itemize}%
\item Find the shortest path that visits~$n$ locations and returns back to its origin.\cite{ABCC2006TTSPACS,LLRKS1985TTSPAGTOCO}%
\end{itemize}%
%
\locateGraphic{}{width=0.4\paperwidth}{\sharedDir/graphics/tsp_china/tsp_china}{0.3}{0.3}%
\end{frame}%
%
%
\def\makebigmarkformat{%
\def\bigmarkformat##1{%
{\alert{\textbf{\Large{##1}}}}%
}}%
\protected\gdef\bigmark#1#2{%
\def\bigmarkformat##1{##1}%
\only<#1>{\makebigmarkformat}%
\bigmarkformat{#2}%
}%
%
\begin{frame}[t]%
\frametitle{Hard Problems~(2)}%
\only<-3,20->{%
\begin{itemize}%
\item These are just two examples from a huge family of problems called~\npHard.%
%
\item<2-> What does that mean?%
\end{itemize}%
}%
%
%
\locate{3-}{\parbox{0.9\paperwidth}{\noindent%
\makeatletter%
\begin{mdframed}[style=@bestPracticeStyle]\Large{%
If an algorithm guarantees to always find the \bigmark{5-6}{best-possible} solution for an \npHard\ problem, then its runtime will \bigmark{9-19}{grow exponential} with the \bigmark{7-8}{input size~\only<-8,20->{\emph{s}}}\only<9-19>{\bigmark{9-19}{\emph{s}}} in the \bigmark{22}{worst case}.%
}\end{mdframed}%
\makeatother%
}}{0.05}{0.7}%
%
\locateGraphic{4-9}{width=0.4\paperwidth}{\sharedDir/graphics/bin_packing/bin_packing}{0.06666666666}{0.25}%
\locateGraphic{4-9}{width=0.4\paperwidth}{\sharedDir/graphics/tsp_china/tsp_china}{0.53333333333}{0.05}%
%
\locate{6}{\parbox{0.4\paperwidth}{\noindent\centering\Large{%
\alert{packing with fewest hauls}%
}}}{0.06666666666}{0.65}%
\locate{6}{\parbox{0.4\paperwidth}{\noindent\centering\Large{%
\alert{shortest round-trip tour}%
}}}{0.53333333333}{0.65}%
%
\locate{8}{\parbox{0.4\paperwidth}{\noindent\centering\Large{%
\alert{number of items to pack}%
}}}{0.06666666666}{0.65}%
\locate{8}{\parbox{0.4\paperwidth}{\noindent\centering\Large{%
\alert{number of cities to visit}%
}}}{0.53333333333}{0.65}%
%
\locate{21}{\parbox{0.7\paperwidth}{\noindent\centering\textbf{\Large{\alert{%
If we want to get the optimal solution,\\it will take too long.%
}}}}}{0.15}{0.45}%
%
\locate{22}{\parbox{0.7\paperwidth}{\noindent\centering\textbf{\Large{%
If we want to get the optimal solution,\\it will \alert{sometimes/often/always} take too long.%
}}}}{0.15}{0.45}%
%
\locateGraphic{10}{width=0.8\paperwidth}{\sharedDir/graphics/function_growth/function_growth_01}{0.1}{0.05}%
\locateGraphic{11}{width=0.8\paperwidth}{\sharedDir/graphics/function_growth/function_growth_02}{0.1}{0.05}%
\locateGraphic{12}{width=0.8\paperwidth}{\sharedDir/graphics/function_growth/function_growth_03}{0.1}{0.05}%
\locateGraphic{13}{width=0.8\paperwidth}{\sharedDir/graphics/function_growth/function_growth_04}{0.1}{0.05}%
\locateGraphic{14}{width=0.8\paperwidth}{\sharedDir/graphics/function_growth/function_growth_05}{0.1}{0.05}%
\locateGraphic{15}{width=0.8\paperwidth}{\sharedDir/graphics/function_growth/function_growth_06}{0.1}{0.05}%
\locateGraphic{16}{width=0.8\paperwidth}{\sharedDir/graphics/function_growth/function_growth_07}{0.1}{0.05}%
\locateGraphic{17}{width=0.8\paperwidth}{\sharedDir/graphics/function_growth/function_growth_08}{0.1}{0.05}%
\locateGraphic{18}{width=0.8\paperwidth}{\sharedDir/graphics/function_growth/function_growth_09}{0.1}{0.05}%
\locateGraphic{19}{width=0.8\paperwidth}{\sharedDir/graphics/function_growth/function_growth_10}{0.1}{0.05}%
%
\end{frame}%
%
%
\begin{frame}[t]%
\frametitle{Traveling Salesperson Problem}%
%
\begin{itemize}%
\item In an \glsFull{optTSP}, we want to find the shortest path through $n$~locations and back to the start\cite{ABCC2006TTSPACS,LLRKS1985TTSPAGTOCO,GP2002TTSPAIV,WCLTTCMY2014BOAAOSFFTTSP,LWLvdBTW2024ATTSPWFFAAHA,WWCTL2016GVLSTIOPSOEAP}.%
\end{itemize}%
%
\locateGraphic{2}{width=0.5\paperwidth}{\sharedDir/graphics/tsp_china/tsp_china}{0.256}{0.22}%
\locateGraphic{3}{width=0.5\paperwidth}{\sharedDir/graphics/tsp_china/tsp_china_solution}{0.256}{0.22}%
%
\locateGraphic{4}{width=0.7\paperwidth}{\sharedDir/graphics/runtime_quality_tradeoff/runtime_quality_tradeoff_1}{0.15}{0.256}%
\locateGraphic{5}{width=0.7\paperwidth}{\sharedDir/graphics/runtime_quality_tradeoff/runtime_quality_tradeoff_2}{0.15}{0.256}%
\locateGraphic{6}{width=0.7\paperwidth}{\sharedDir/graphics/runtime_quality_tradeoff/runtime_quality_tradeoff_3}{0.15}{0.256}%
\locateGraphic{7}{width=0.7\paperwidth}{\sharedDir/graphics/runtime_quality_tradeoff/runtime_quality_tradeoff_4}{0.15}{0.256}%
\locateGraphic{8}{width=0.7\paperwidth}{\sharedDir/graphics/runtime_quality_tradeoff/runtime_quality_tradeoff_5}{0.15}{0.256}%
\locateGraphic{9}{width=0.7\paperwidth}{\sharedDir/graphics/runtime_quality_tradeoff/runtime_quality_tradeoff_6}{0.15}{0.256}%
\locateGraphic{10}{width=0.7\paperwidth}{\sharedDir/graphics/runtime_quality_tradeoff/runtime_quality_tradeoff_7}{0.15}{0.256}%
\locateGraphic{11}{width=0.7\paperwidth}{\sharedDir/graphics/runtime_quality_tradeoff/runtime_quality_tradeoff_8}{0.15}{0.256}%
\end{frame}%
%
\section{Randomly Guessing Solutions}%
%
\begin{frame}%
\frametitle{\emph{Almost} Solving Hard Problems}%
\begin{itemize}%
\item For many problems, it is very easy to \inQuotes{guess} a solution.%
%
\item<2-> This solution will (very likely) not be optimal.%
%
\item<3-> It will very likely be very bad.%
%
\item<4-> But it will be better than nothing.%
\end{itemize}%
\end{frame}%
%
\begin{frame}[t]%
\frametitle{Random Sampling for the TSP}%
\begin{itemize}%
\only<-3>{%
\item Back to the \glsFull{optTSP}.%
%
\item<2-> How do we \inQuotes{guess} a solution?%
}%
%
\item<3-> We could write down the cities in a random order.%
%
\only<-13>{%
\item<9-> If we do this, we will get a bad result.%
%
\item<10-> But we will get it quickly.%
%
\item<11-> And we can do this often.%
%
\item<12-> For this small problem, my computer can do this \emph{millions} of times in a few seconds.%
%
\item<13-> We can let the algorithm run for as much time as we can wait {\dots} and then simply take the best random tour we found.%
}%
%
\only<-14>{%
\item<14-> This is the improvement of the best-so-far solution of this so-called Random Sampling algorithm over the number of solutions it has generated and tested~(the so-called \glsFullpl{objFE}).%
}%
%
\only<-15>{%
\item<15-> This is the improvement of the best-so-far solution of this so-called Random Sampling algorithm over the consumed runtime in milliseconds.%
}%
%
\item<16-> And for this small problem, this random sampling eventually finds the optimal tour.%
%
\end{itemize}%
%
\locateGraphic{4}{width=0.5\paperwidth}{\sharedDir/graphics/tsp_china_examples/rs/tsp_china_example_rs_1}{0.25}{0.26}%
\locateGraphic{5}{width=0.5\paperwidth}{\sharedDir/graphics/tsp_china_examples/rs/tsp_china_example_rs_2}{0.25}{0.26}%
\locateGraphic{6}{width=0.5\paperwidth}{\sharedDir/graphics/tsp_china_examples/rs/tsp_china_example_rs_3}{0.25}{0.26}%
\locateGraphic{7}{width=0.5\paperwidth}{\sharedDir/graphics/tsp_china_examples/rs/tsp_china_example_rs_4}{0.25}{0.26}%
\locateGraphic{8-11}{width=0.5\paperwidth}{\sharedDir/graphics/tsp_china_examples/rs/tsp_china_example_rs_5}{0.25}{0.26}%
%
\locateGraphic{14}{width=0.625\paperwidth}{\sharedDir/graphics/tsp_china_examples/evaluation/progress_over_log_FEs_rs}{0.1875}{0.37}%
\locateGraphic{15}{width=0.625\paperwidth}{\sharedDir/graphics/tsp_china_examples/evaluation/progress_over_log_ms_rs}{0.1875}{0.37}%
%
\end{frame}%
%
\begin{frame}%
\frametitle{Metaheuristic Optimization}%
%
\begin{center}%
\resizebox{0.77\paperwidth}{!}{\parbox{0.5\paperwidth}{\noindent%
\alert{Principle~1}:~We can randomly construct solutions.\uncover<2->{\\[10pt]\noindent%
\alert{Problem~1}:~Guessing an optimal (or even just good) solution randomly is very unlikely.%
}%
}}%
\end{center}%
%
\end{frame}%
%
\begin{frame}%
\frametitle{Why don't we use the \gls{algoAstar} for the TSP?}%
%
\begin{itemize}%
\item Before, we talked about the \gls{algoAstar} for path finding.%
%
\item<2-> Isn't the \glsShort{optTSP} about finding a path?%
%
\item<3-> Why do we not use it for the \glsShort{optTSP}?%
%
\item<7-> \alert<7>{It will crash for any larger \glsShort{optTSP}.}%
%
\item<8-> But it can be done and used for our small \glsShort{optTSP}.%
%
\item<9-> So let's do it.\uncover<10->{ (I spare you all details.)}%
%
\end{itemize}%
%
\bigskip%
%
\uncover<4->{%
\begin{enumerate}%
\item The \gls{algoAstar} needs \alert<7>{memory exponential} in the number of nodes in the final solution\cite{SE:SO:WITCOAEIM}.\uncover<5->{ %
Which is the number of cities in the \glsShort{optTSP}\dots%
}%
%
\item<6-> Also, it is a bit awkward: What are the heuristic and the goal state?%
\end{enumerate}%
}%
\end{frame}%
%
\begin{frame}%
\frametitle{Why don't we use the \gls{algoAstar} for the TSP?}%
%
\locateGraphic{1}{width=0.8\paperwidth}{\sharedDir/graphics/tsp_china_examples/evaluation/progress_over_log_ms_astar}{0.1}{0.2}%
\locateGraphic{2-}{width=0.55\paperwidth}{\sharedDir/graphics/tsp_china_examples/evaluation/progress_over_log_ms_astar}{0.4}{0.35}%
%
\uncover<2->{\parbox{0.4\paperwidth}{\noindent%
\begin{itemize}%
\item Because not only will it use too much memory for larger instances\dots%
%
\item<3-> {\dots}it is also slower than randomly guessing on smaller instances.%
\end{itemize}%
}}%
%
\end{frame}%
%
\begin{frame}%
\frametitle{Random Sampling still is a bad algorithm}%
%
\uncover<-3,5->{%
\begin{itemize}%
\item Random sampling is a bad algorithm for the \glsShort{optTSP}~(and basically all other problems, too).%
%
\item<2-> It relies on randomly guessing good tours.%
%
\item<3-> If we have $n$~cities, then there are $1*2*3*\dots*(n-1)*n=\factorial{n}$ tours.%
%
\item<5-> Factorial growth~(\factorial{n}) is even worse than exponential growth\dots%
%
\item<6-> So if we try to randomly sample the best possible tour, our chance is extremely small\dots%
\end{itemize}%
}%
%
\locateGraphic{4}{width=0.9\paperwidth}{\sharedDir/graphics/function_growth/function_growth_11}{0.05}{0.175}%
\end{frame}%
%
\section{Local Search:~Using Information}%
%
\begin{frame}%
\frametitle{Random Sampling is a bad algorithm}%
\begin{itemize}%
\item In each step, random sampling creates a completely random tour.%
\item<2-> It does not gather any information.%
\item<3-> It keeps the best tour, but does not use any information inside this tour.%
\item<4-> Just guessing answers randomly is not a good method.%
\item<5-> Clearly, seeing millions of tours, we should be able learn \emph{something} and somehow use that to find better tours???%
\end{itemize}%
\end{frame}%
%
\begin{frame}%
\frametitle{Randomized Local Search}%
%
\begin{itemize}%
\item But what could we do?%
%
\item<2-> \inQuotes{\alert<2>{In the neighborhood of a solution, there will probably be better or worse solutions.}}%
%
\item<3-> neighborhood $=$ solutions that are similar.%
%
\item<4-> If we have a tour~$x$, we could randomly pick two cities in the tour and swap them.%
%
\item<5-> If we do this, maybe we could get a better tour or a worse tour.%
%
\item<6-> We could keep a better tour, but throw away a worse one.%
%
\item<7-> Then we do the same with a better tour that we find, and with yet a better tour, and so on.%
%
\item<8-> This method is called \glsFull{algoRLS}.%
\end{itemize}%
\end{frame}%
%
\begin{frame}%
\frametitle{Randomized Local Search for the TSP}%
%
\locateGraphic{1,4}{width=0.76\paperwidth}{\sharedDir/graphics/tsp_china_examples/rls2/tsp_china_example_rls2_01_04}{0.12}{0.08}%
\locateGraphic{2}{width=0.76\paperwidth}{\sharedDir/graphics/tsp_china_examples/rls2/tsp_china_example_rls2_02}{0.12}{0.08}%
\locateGraphic{3}{width=0.76\paperwidth}{\sharedDir/graphics/tsp_china_examples/rls2/tsp_china_example_rls2_03}{0.12}{0.08}%
\locateGraphic{5}{width=0.76\paperwidth}{\sharedDir/graphics/tsp_china_examples/rls2/tsp_china_example_rls2_05}{0.12}{0.08}%
\locateGraphic{6,9}{width=0.76\paperwidth}{\sharedDir/graphics/tsp_china_examples/rls2/tsp_china_example_rls2_06_09}{0.12}{0.08}%
\locateGraphic{7}{width=0.76\paperwidth}{\sharedDir/graphics/tsp_china_examples/rls2/tsp_china_example_rls2_07}{0.12}{0.08}%
\locateGraphic{8}{width=0.76\paperwidth}{\sharedDir/graphics/tsp_china_examples/rls2/tsp_china_example_rls2_08}{0.12}{0.08}%
\locateGraphic{10}{width=0.76\paperwidth}{\sharedDir/graphics/tsp_china_examples/rls2/tsp_china_example_rls2_10}{0.12}{0.08}%
\locateGraphic{11,14}{width=0.76\paperwidth}{\sharedDir/graphics/tsp_china_examples/rls2/tsp_china_example_rls2_11_14}{0.12}{0.08}%
\locateGraphic{12}{width=0.76\paperwidth}{\sharedDir/graphics/tsp_china_examples/rls2/tsp_china_example_rls2_12}{0.12}{0.08}%
\locateGraphic{13}{width=0.76\paperwidth}{\sharedDir/graphics/tsp_china_examples/rls2/tsp_china_example_rls2_13}{0.12}{0.08}%
\locateGraphic{15}{width=0.76\paperwidth}{\sharedDir/graphics/tsp_china_examples/rls2/tsp_china_example_rls2_15}{0.12}{0.08}%
\locateGraphic{16}{width=0.76\paperwidth}{\sharedDir/graphics/tsp_china_examples/rls2/tsp_china_example_rls2_16}{0.12}{0.08}%
\locateGraphic{17}{width=0.76\paperwidth}{\sharedDir/graphics/tsp_china_examples/rls2/tsp_china_example_rls2_17}{0.12}{0.08}%
\locateGraphic{18}{width=0.76\paperwidth}{\sharedDir/graphics/tsp_china_examples/rls2/tsp_china_example_rls2_18}{0.12}{0.08}%
\locateGraphic{19}{width=0.76\paperwidth}{\sharedDir/graphics/tsp_china_examples/rls2/tsp_china_example_rls2_19}{0.12}{0.08}%
\locateGraphic{20}{width=0.76\paperwidth}{\sharedDir/graphics/tsp_china_examples/rls2/tsp_china_example_rls2_20}{0.12}{0.08}%
\end{frame}%
%
\begin{frame}%
\frametitle{Randomized Local Search for the TSP}%
\locateGraphic{}{width=0.8\paperwidth}{\sharedDir/graphics/tsp_china_examples/evaluation/progress_over_log_ms_rls2}{0.1}{0.2}%
\end{frame}%
%
\begin{frame}%
\frametitle{Randomized Local Search for the TSP}%
\begin{itemize}%
\item \glsShort{algoRLS} is much faster than random sampling.%
%
\item<2-> However\only<-2>{\dots}\uncover<3->{, if our algorithm is at a tour~$x$, then it can only find other tours by swapping two cities.}%
%
\item<4-> There are some tours that it cannot reach in one step.%
%
\item<5-> If all tours that we can reach in one step are worse than~$x$, we will never leave~$x$.%
%
\item<6-> If~$x$ is not the optimum, then we are stuck and won’t ever find the optimal/best tour.%
\end{itemize}%
\end{frame}%
%
\begin{frame}%
\frametitle{Metaheuristic Optimization}%
%
\begin{center}%
\resizebox{0.77\paperwidth}{!}{\parbox{0.5\paperwidth}{\noindent%
\alert{Principle~2}:~Applying a small random change to an existing solution sometimes can give us a better solution.%
\uncover<2->{ This way, we can step-by-step get better solutions.}\uncover<3->{\\[10pt]\noindent%
\alert{Problem~2}:~Sometimes, a small change is not enough.\uncover<4->{ We can get stuck at a so-called \emph{local} optimum.}%
}%%
}}%
\end{center}%
%
\end{frame}%
%
\begin{frame}%
\frametitle{Randomized Local Search for the TSP}%
\begin{itemize}%
%
\item \alert{If all tours that we can reach in one step are worse than the current tour~$x$, then our \glsShort{algoRLS} will never leave~$x$.}%
%
\item<2-> There are many things that we can do\only<-2>{.}\uncover<3->{, for example:%
\begin{enumerate}%
\item We could restart the algorithm at a new starting point after some time.%
%
\item<4-> \alert{Or we could sometimes allow it swap more cities in the current tour.}%
\end{enumerate}%
}%
\end{itemize}%
\end{frame}%
%
%
\begin{frame}%
\frametitle{Randomized Local Search with a Larger Neighborhood}%
\locateGraphic{1,4}{width=0.76\paperwidth}{\sharedDir/graphics/tsp_china_examples/rlsn/tsp_china_example_rlsn_01_04}{0.12}{0.08}%
\locateGraphic{2}{width=0.76\paperwidth}{\sharedDir/graphics/tsp_china_examples/rlsn/tsp_china_example_rlsn_02}{0.12}{0.08}%
\locateGraphic{3}{width=0.76\paperwidth}{\sharedDir/graphics/tsp_china_examples/rlsn/tsp_china_example_rlsn_03}{0.12}{0.08}%
\locateGraphic{5}{width=0.76\paperwidth}{\sharedDir/graphics/tsp_china_examples/rlsn/tsp_china_example_rlsn_05}{0.12}{0.08}%
\locateGraphic{6,9}{width=0.76\paperwidth}{\sharedDir/graphics/tsp_china_examples/rlsn/tsp_china_example_rlsn_06_09}{0.12}{0.08}%
\locateGraphic{7}{width=0.76\paperwidth}{\sharedDir/graphics/tsp_china_examples/rlsn/tsp_china_example_rlsn_07}{0.12}{0.08}%
\locateGraphic{8}{width=0.76\paperwidth}{\sharedDir/graphics/tsp_china_examples/rlsn/tsp_china_example_rlsn_08}{0.12}{0.08}%
\locateGraphic{10}{width=0.76\paperwidth}{\sharedDir/graphics/tsp_china_examples/rlsn/tsp_china_example_rlsn_10}{0.12}{0.08}%
\locateGraphic{11}{width=0.76\paperwidth}{\sharedDir/graphics/tsp_china_examples/rlsn/tsp_china_example_rlsn_11}{0.12}{0.08}%
\locateGraphic{12}{width=0.76\paperwidth}{\sharedDir/graphics/tsp_china_examples/rlsn/tsp_china_example_rlsn_13}{0.12}{0.08}%
\locateGraphic{13}{width=0.76\paperwidth}{\sharedDir/graphics/tsp_china_examples/rlsn/tsp_china_example_rlsn_14_19}{0.12}{0.08}%
\locateGraphic{14}{width=0.76\paperwidth}{\sharedDir/graphics/tsp_china_examples/rlsn/tsp_china_example_rlsn_15}{0.12}{0.08}%
\locateGraphic{15}{width=0.76\paperwidth}{\sharedDir/graphics/tsp_china_examples/rlsn/tsp_china_example_rlsn_16}{0.12}{0.08}%
\locateGraphic{16}{width=0.76\paperwidth}{\sharedDir/graphics/tsp_china_examples/rlsn/tsp_china_example_rlsn_20}{0.12}{0.08}%
\locateGraphic{17}{width=0.76\paperwidth}{\sharedDir/graphics/tsp_china_examples/rlsn/tsp_china_example_rlsn_21}{0.12}{0.08}%
\end{frame}%
%
\begin{frame}%
\frametitle{Randomized Local Search with a Larger Neighborhood}%
\locateGraphic{}{width=0.8\paperwidth}{\sharedDir/graphics/tsp_china_examples/evaluation/progress_over_log_ms_rlsn}{0.1}{0.2}%
\end{frame}%
%
\begin{frame}%
\frametitle{Randomized Local Search with a Larger Neighborhood}%
{\color{gray}{%
\begin{itemize}%
%
\item If all tours that we can reach in one step are worse than the current tour~$x$, then our \glsShort{algoRLS} will never leave~$x$.%
%
\item There are many things that we can do, for example:%
\begin{enumerate}%
\item We could restart the algorithm at a new starting point after some time.%
%
\item Or we could sometimes allow it swap more cities in the current tour.%
\end{enumerate}%
%
\item<2-> \alert{The larger neighborhood of solutions that can be reached in one step makes the search slower, but allows it to escape from local optima.}%
%
\end{itemize}}}%
\end{frame}%
%
\begin{frame}%
\frametitle{Metaheuristic Optimization}%
%
\only<-2>{%
\begin{center}%
\resizebox{0.77\paperwidth}{!}{\parbox{0.5\paperwidth}{\noindent%
\alert{Principle~3}:~Larger changes can help us escape from local optima.\uncover<2->{\\[10pt]\noindent%
\alert{Problem~3}:~Larger changes are less likely to yield improvements (because the new solution is more different), so they slow down the search.%
}%
}}%
\end{center}%
}%
\only<3->{%
\begin{center}%
\resizebox{0.77\paperwidth}{!}{\parbox{0.5\paperwidth}{\noindent%
There are many more principles and problems surrounding (metaheuristic) optimization.\uncover<4>{\\[10pt]%
But we will leave it at what we have seen so far.%
}%
}}%
\end{center}%
}%
%
\end{frame}%
%
\section{Summary}%
%
\begin{frame}[t]%
\frametitle{Summary on Algorithms}%
\begin{itemize}%
\item We have learned some basic algorithmic principles.%
%
\item<2-> Current research tries to improve both the speed of algorithms as well as the solution quality.%
%
\item<3-> This requires carefully balancing the step-size of algorithms and to develop methods against getting stuck at local optima.%
\end{itemize}%
\end{frame}%
%
\begin{frame}[t]%
\frametitle{Summary on the TSP}%
\begin{itemize}%
\item Today, we can actually solve \glsShort{optTSP}s with tens of thousands of cities to optimality\cite{ABCC2006TTSPACS,C2013WT}.\uncover<5->{~\alert{[but not all of them!]}}%
%
\item<2-> For this, we actually use principles similar to the \gls{algoAstar}, just a bit differently, in methods like branch and bound\cite{ABCC2006TTSPACS,LMSK1963AAFTTSP,CEG2005CWDPIFT}.%
%
\item<3-> We can get close-to-optimal solutions for \glsShort{optTSP}s with millions of cities.\uncover<6->{~\alert{[close-to-optimal, but not optimal~(at least not always)!]}}%
%
\item<4-> For this, we use algorithms a bit similar to \glsFull{algoRLS}, but with more targeted search steps, e.g., in the Lin-Kernighan-Helsgaun algorithm\cite{H2009GKOSFTLKTH}.%
%
\end{itemize}%
\end{frame}%
%
\begin{frame}%
\frametitle{Summary}%
\begin{itemize}%
\item Today, we discussed what optimization is.%
%
\item<2-> Today, you also saw some of the basic principles and methods that are inside of the algorithms used in optimization and operations research.%
%
\item<3-> You can imagine that the same principles that we tested on the \glsShort{optTSP} will work on many different kinds of problems.%
%
\item<4-> For example:~As long as we can randomly construct and randomly modify a solution, we can attack the problem with \glsFull{algoRLS}.%
%
\item<5-> Understanding the basic principles of optimization is not very hard.%
%
\end{itemize}%
\end{frame}%
%
\section{Advertisement}%
%
\begin{frame}[t]%
\frametitle{Programming with Python}%
We have a freely available course book on \citetitle{programmingWithPython} at \citeurl{programmingWithPython}, with focus on practical software development using the \python\ ecosystem of tools\cite{programmingWithPython}.%
%
\locateGraphic{}{width=0.63\paperwidth}{\sharedDir/graphics/advertisement/snippets/programmingWithPythonSnippet}{0.025}{0.3}%
\locateGraphic{}{width=0.27\paperwidth}{\sharedDir/graphics/advertisement/urlQr/programmingWithPythonCourseUrl}{0.675}{0.4}%
%
\end{frame}%
%
\begin{frame}[t]%
\frametitle{Databases}%
We have a freely available course book on \citetitle{databases} at \citeurl{databases}, with actual practical examples using a real \dbms\cite{databases}.%
%
\locateGraphic{}{width=0.63\paperwidth}{\sharedDir/graphics/advertisement/snippets/databasesSnippet}{0.025}{0.3}%
\locateGraphic{}{width=0.27\paperwidth}{\sharedDir/graphics/advertisement/urlQr/databasesCourseUrl}{0.675}{0.4}%
%
\end{frame}%
%
%
\endPresentation%
\end{document}%%
\endinput%
%
